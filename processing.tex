\section{Processing of data}
\protect\label{section:processing}

In this section the processing of the data on a daily and ongoing basis is
considered.

\subsection{Daily processing}
\protect\label{section:dailyproc}

A daily routine was set up on the Cluster (and in parallel on my home machine)
to do the following.

\begin{enumerate}
  \item Load rows of data corresponding to the observations taken for the
  preceding night.
  \item For the observations of interest, in terms of the targets and for the
  visible light filters, load the corresponding FITS files. Also note parameters
  such as moon phase and distance, exposure time etc.
  \item Load rows of data corresponding to the daily bias and flat files.
  \item Load corresponding FITS files where this is useful (i.e. excluding
  those with a gain of other than 1)
  \item \textit{Not done yet but about to add} note information about
  reliability of each image, assigning a ``quality'' value between 0 and 1.
  Currently this is ``binary'' only but some images, perhaps partly obscured by
  clouds, can still be partly used with a suitable weighting. \textit{I have
  already written the python interface to select images in a given range of
  quality}.
  \item \textit{Not done yet but about to add} do outline find of objects in
  each usable image.
\end{enumerate}

\subsection{Brightness plot of images}
\protect\label{section:recordadus}

After taking each usable image and with the locations of available objects
already found, it is possible to make a light curve for the object over time by
reference to a common subset of reference objects.

Unfortunately due to the differing orientations of the images, the subset of
reference objects is Small the more images are taken. I have set up to be able
to select images by orientation, using 4 different selections may be chosen, 0
representing an orientation with increasing Declination along the Y axis and 1
to 3 respectively selecting an anti-clockwise 90\degree turn from that. This
increases the chances of being able to find a common set of reference objects.
\textit{I need to illustrate this} as does limiting selections to the brightest
overall images.

As discussed in Section \ref{section:fwoptap}, the brightness and variability of
reference objects needs to be established so each frame can be evaluated on the
basis of the reference objects available.

\clearpage