\appendix 
\section{Database and interfaces}
\protect\label{section:database}

In order, to facilitate the analysis of the data, I created a database
constructed to be able to retrieve given observations or other data according to
criteria as required, together with a set of Python interfaces, using
\textit{Numpy} and other facilities including routines from \textit{Astropy}.

The tables in this database are as follows.

\begin{description}
\item[obsinf] Information about observations.
\item[iforbinf] Information about individual flat and bias files.
\item[forbinf] Information about master flat and bias files.
\item[objdata] Details of objects in the vicinity of target objects.
\item[objalias] Alternative names for objects in \textbf[objdata].
\item[objpm] Records of positions of objects with appreciable proper motions.
\item[findresult] Notes of objects found in each observation.
\item[fitsfile] Copies of the FITS files for observations in active
consideration.
\end{description}

Not all of the FITS files are stored; only those with the visible light filters
and of the target {\rdwarf} objects are saved. The Python routines are designed
to transparently either fetch a FITS file from the database \textbf[fitsfile]
table or fetch from the Red Dots project database (with optional local
caching) as required.

\subsection{The obsinf table}
\protect\label{section:obsinftable}

The \textbf{obsinf} table records details of observations taken from the
relevant FITS file (which may be saved locally in the \textbf{fitsfile} table).

Columns in this database give the following information, and a Python routine
enables a selection of records of observations to be retrieved according to
ranges of these criteria.

\begin{description}
\item[obsind] A unique integer used for internal cross-reference.
\item[serial] The serial number of the observation at the Red Dots project.
\item[date\_obs] The date and time of the observation.
\item[object] The target object name
\item[filter] The filter.
\item[median] Median pixel value.
\item[mean] Mean pixel value.
\item[minv] Minimum pixel value.
\item[maxv] Maximum pixel value.
\item[std] Standard deviation of pixel value.
\item[skew] Skewness of pixel value.
\item[kurt] Kurtosis of pixel value.\footnote{Skewness and kurtosis are only
recorded as these are used in construction of the master flat files and this is
included for consistency with \textbf{iforbinf}.}
\item[moonphase] Phase of moon as percentage.
\item[moondist] Angular distance of moon from target.
\item[airmass] Airmass to target,
\item[quality] Quality assessment of observation, between 0 (rejected) and 1
(best).
\item[exptime] Exposure time.
\end{description}

\subsection{The iforbinf table}
\protect\label{section:iforbinftable}

The \textbf{iforbinf} table records details of individual flat or bias files
taken from the relevant FITS file (which may be saved locally in the \textbf{fitsfile} table).

Columns in this database give the following information, and a Python routine
enables a selection of records of the file to be retrieved according to
ranges of these criteria.

\begin{description}
\item[iforbind] A unique integer used for internal cross-reference.
\item[typ] Type of file, \texttt{flat}, \texttt{bias} or \texttt{dark}.
\item[serial] The serial number of the file at the Red Dots project.
\item[date\_obs] The date and time for taking the file.
\item[filter] The filter (only \texttt{g}, \texttt{i}, \texttt{r} or
\texttt{z}).
\item[median] Median pixel value.
\item[mean] Mean pixel value.
\item[minv] Minimum pixel value.
\item[maxv] Maximum pixel value.
\item[std] Standard deviation of pixel value.
\item[skew] Skewness of pixel value.
\item[kurt] Kurtosis of pixel value.\footnote{Skewness and kurtosis are only
recorded as these are used in construction of the master flat files.}
\end{description}

\subsection{The objdata table}
\protect\label{section:objdatatable}

The \textbf{objdata} table records details of objects in the vicinity of one of
the target objects).

Columns in this database give the following information, and a Python routine
enables objects in the vicinity of a given target to be retrieved, making
appropriate allowances for proper motions (in conjunction with the
\textbf[objpms] table).

\begin{description}
\item[ind] A unique integer used for internal cross-reference.
\item[objname] An internal name for the object (short names such as
\textbf{GJ551} for {\prox} are preferred).
\item[dispname] A name to be used for displays. Latex format is used for easy
insertion of results into Latex reports such as this, for example {\prox} is
stored as \texttt{$\backslash$prox}.
\item[objtype] Type of object, such as \texttt{Star}.
\item[radeg] \texttt{J2000}-based Right Ascension in degrees.
\item[decdeg]\texttt{J2000}-based Declination in degrees.
\item[rapm] Proper motion in Right Ascension, in milli-arcseconds / year.
\item[decpm]Proper motion in Declination, in milli-arcseconds / year.
\item[apsize]Optimised aperture size, possibly fractional.
\end{description}

This is combined with various brightness information, in various bands, together
with standard deviations. (These fields are currently not finalised).

\clearpage
