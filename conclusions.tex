\section{Further work}
\protect\label{section:worktocome}

From the above, it is possible to draw the following program for further work.

\begin{enumerate}
  \item Hotspot and bad pixel workarounds need to be created to eliminate
  ``false positive'' identifications and minimise the negative pixels after
  subtraction of the bias files. This is particularly important for the
  \texttt{g} filter, but also necessary for the other filters.
  \item The is a need to fine-tune the apertures to correctly select objects
  and also to reject small bright points.
  \item Need to check daily flat field images for features which have been noted
  on some \textit{I need to give example}.
  \item Need to fine-tune edge ``trimming'' with appropriate values for each
  filter.
  \item Need to build a database of objects and magnitudes for each filter so as
  to be able to reliably use all the images regardless of the patch of sky in
  use.
  \item An attempt to merge results for all filters on a given day.
  \item Better treatment of the background sky level. Possible avenues for improvement are suggested by using the
      techniques in \citet{dubovsky17}.
  \item Being less conservative in acceptance of reference objects.
  \item Interpolation of the ADU counts of reference objects outside the image
  frame.
\end{enumerate}

