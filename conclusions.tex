\section{Further work}
\protect\label{section:worktocome}

The following is the programme of work I need to undertake to ready the project
to a state where serious science can be achieved.

In all cases, the work to be done will need revisiting and refining and
``iterating until convergence'' achieved.

\subsection{Completion and improvement of work to date}
\protect\label{section:fwimprovecompletion}

This currently documents work in progress and the following needs to be achieved
or at least an initial version complete before the project is ready to move
forward.

\subsubsection{Completion of revised calibration}
\protect\label{section:fwcalib}

I need to put together the work on revised bias and flat files, also of bad
pixel identification, so as to minimise the loss of useful data, refining whether
to discard a pixel altogether or assign a low weighting.

Some parts, especially of limits to flat fields, may not be needed; there is
plenty of scope for refinement.

Also needed is a reasonable assessment of the uncertainty of any measurements
taken.

I need to insert new versions of Fig. \ref{fig:initgexample} and following
into this report in section \ref{section:postcalibration} showing improvements
made.in these fields, with a first version ready by the end of the second week
in June 2022.

\subsubsection{Revise identification of objects other than targets}
\protect\label{section:fwtargets}

I populated the database of objects from SIMBAD and GAIA EDR3 in May 2021 and
this needs revising. Some of the objects are closer to each other than the
resolution of the telescope, about 0.6 arc-seconds per pixel, can distinguish.

The identification of close objects is currently slightly random due to the
different orientations of the images and the order in which the search
algorithm takes sources. It may be that there are larger common subsets of
reference objects than are currently apparent. It may well be feasible to
regard close objects as a composite object for the purposes of using as
reference objects.

This will need doing with some care and I would hope to be confident of the
results for at least {\prox} by the end of June 2022.

\subsubsection{Optimise apertures and determine magnitudes and variability of
reference objects} \protect\label{section:fwoptap}

In order to evaluate magnitudes over all the images, it is necessary to have
confidence in the reference objects available in all of the images, rather than
the common subset of objects which appears in all images, which becomes
vanishingly small, especially for {\ross} and {\bstar} and for the \texttt{i}
and \texttt{z} filters.

This will need running through and comparing each object with each other object,
this is more of a processing issue than anything.

It would be desirable to build a picture of variability and compare this with the
variability statistics from other sources, so that an uncertainty figure can
be made for each target.

Also needed is a standard optimised aperture size for calculation of the
brightness of objects in any given image.

Although this is a highly iterative process, I should hope to have a ``first
pass" of this complete by mid-August 2022.

\subsection{Extraction of data}
\protect\label{section:fwextract}

The whole object is to obtain believable light curves and observational data
from the REM image data including values for uncertainty in which it is possible
to have confidence and this is what I intend to present by the end of September
2022.

Obtaining periodic data is particularly important and a clear objective of this
study would be to try to obtain this for \ross, (and consequently the
inclination angle) although as discussed in Section \ref{section:introross}. with a maximum
period of $3.5 \pm 1.5$ days, but with observation days months apart as shown in Fig. \ref{fig:rdwarfhist}, there will
be a considerable number of aliases to contend with.

Obviously too, an increase in the understanding of activity cycles of all three
main targets will hopefully be a useful consequence of this.

\subsection{Calculation of Barycentric dates, periodograms}
\protect\label{section:fwbarycentric}

Barcycentric dates need to be computed and inserted into the observation records
before periodograms can be extracted. Then need to start producing periodograms
of the data. \textit{I have done this.}

\subsection{Variations and improvements}
\protect\label{section:fwvariations}

There are a substantial number of choices made in the analysis of the data and
scope for seeing whether improvements in the accuracy can be brought about by
tuning some of them. In mind are:

\begin{enumerate}
  \item Using a weighted sum in calculating peaks as opposed to summing the ADUs
  and calculating the error.
  \item Using sub-standard images with weightings.
  \item Experiment with weightings of areas round the edges of the images, where
  vignetting and other effects affect the exposures.
  \item Using different sized apertures with different filters.
  \item Selection of different reference objects with different filters.
  \item Asking for changes to the exposure times.
  \item Plotting of periods of other objects found in the frames.\footnote{It
  might be necessary to revise the Barycentric dates in these cases but the
  ones for the targets will probably be accurate enough, but must check.}
\end{enumerate}

\subsection{Incorporation of other data}
\protect\label{section:fwincorp}

There is a complete set of REMIR data taken simultaneously with all the other
options, with calibration already done. It would be wasteful not to make
maximums use of this for comparison and for incorporation into the results where
possible.

Reference object sets are a little more limited with these as the {\rdwarf}
targets are so much brighter in the Infrared compared to the others.

I also intend to look at other datasets and where possible undertake a similar
task with these.

\clearpage
