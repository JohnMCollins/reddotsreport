\section{Analysis and conclusions}
\protect\label{section:analysis}

From the above, it is possible to draw the following conclusions and concerns:

\subsection{Accuracy}
The accuracy of the ADU counts in many places is of concern; the sky and/or bias
counts are around 300 in all places and the maximum count in the target pixels
can be as low as 3,800 and in some cases, as with the less bright objects which
might be used for reference the ADU count can be of the order of 400.

Very few of the images for \texttt{i} and \texttt{z} filters have anything
detectable at all in apart from the target.

It would be good to increase the exposure times, which should improve the
accuracy in all cases, but unfortunately some of the \texttt{i} filter pixels in
some of the images are close to saturation, as shown in Table
\ref{table:pmaxima}.

\subsection{Master flat files}
It is hard to feel confident in the accuracy of the master flat files given that
the daily flat files which go up to make them very so widely (see Table
\ref{table:flatdist}). Also, there is the suspicion that some of these files are
saturated or nearly so.

The master flat files are not normalised as advertised. \textbf{This
invalidates most of the already prepared results, which I'll have to redo.}

\subsection{Analysis improvements}
\protect\label{section:analimprove}

Owing to the failure to find reference objects in many cases, such as all the infrared bands and two out of the four
visible light bands, it would seem desirable to improve the analysis of images especially where only the target is
visible. The following seem to be areas for improvement of the analysis.

\begin{enumerate}
  \item The daily observations need to be appropriately binned and corrected for
  air mass.
  \item An attempt to merge results for all filters on a given day.
  \item Better treatment of the background sky level. Possible avenues for improvement are suggested by using the
      techniques in \citet{dubovsky17}.
  \item Being less conservative in acceptance of reference objects.
  \item Interpolation of the ADU counts of reference objects outside the image
  frame.
  \item Aperture sizes for reference objects needs to be fine-tuned.
\end{enumerate}

\subsection{REM system suggestions}
\protect\label{section:remimprove}

Perhaps it would be possible to increase the exposure time for the \texttt{g}
and \texttt{r} filters from 5 to 20 seconds. It would be possible to double the
exposure time for the \texttt{z} filter. These changes would improve the SNR for
measurements based on these. It would not be possible to increase the exposure
time for the \texttt{i} filter as some of the images have pixel values
approaching saturation already.

Perhaps wonder if it is sensible to take the bias values at 11am in the morning
when the observations are all taken around 3am and possibly the CCD is at a different
temperature.

