\section{Introduction}
\protect\label{section:intro}

\engdate

This report is an analysis of the observational data from the REM (Rapid Eye
Mount) camera in La Silla, Chile, operated by the Red Dots Project, This work
commencing in March 2015. \footnote{See \texttt{https://reddots.space/}}. In
this project a series of observations are made of a patch of sky in the vicinity of particular target using four visual
light filters on the ROS2 telescope and 3 infrared filters on the REMIR
telescope. The images from the visbile light telescope are 1024 by 1-24 pixels, although
the effective area is only 900 by 900 and the images from the infrared are 512
by 512 pixels. In each case a 16-bit CCD records the counts for exach pixel.
Flat field and bias frames are only available for the visible light cases.

The primary targets are \rdwarf s, although observations are made of other types
of star during the course of each night.In this report, only the {\rdwarf} observations are considered and only the
visible light cases.

\subsection{Targets}

The three main targets of the REM observations are \prox, {\bstar} and Ross 154,
all \rdwarf s of spectral types M6, M4 and M3.5 respectively and 4.2, 6 and 9.6
ly distant, putting them at nearest, fourth nearest and eleventh nearest known
stellar objects to the solar system.The main subject of study is the
identification of periodic signals in the light curves.

\subsubsection{\prox}
{\prox} is of perennial interest as the closest star to the solar system, at 4.2 ly. It is of spectral type M5.5.
Despite the interest and available data, calculations of the the rotation period
of {\prox} have varied over the years.  Studies have reported periods
ranging from the $ 31.5 \pm 1.5 $ days of \citet{guinan96}, the 41.3 days of \citet{benedict93}
and s between 82 and 84 days in \citealt{benedict92,benedict98}.  \citet{kurster99} found that the period is not less
than 50 days, whilst more recently \citet{kiraga07} reported a value of 82.5
days. All these estimates were based on photometric measurements \citet[Table 3]{suarezmascareno15} reported a value of 116.6 days, using a
measurement of {\ha}. In \citet{collins17} the rotation periods was calculated
ast 82.6 $\pm$ 0.1 days, based primarily on photometric evidence, with some
support from spectroscopic analysis of an {\ha} peak.

In 2016 a planet of 1.3 Earth mass was reported in the habitable
zone of \prox, known as {\prox} b. \citep{angladaescude16}. Its orbital period was calculated as just over 11
days. The following year further studies have suggested dust belts \citep{anglada17} but yet furhter
studies have suggested that a large flare affected those results \citep{macgregor18}.

THe frequent flare activity is unusual in a star of slow rotation like \prox,
previous studies such as In \citet{mohanty03} setting out the correlation
between the projected otational velocity {\vsini} and the
activity in mid-M to L-dwarfs, In \citet{vida19} the flaring activity was
calculated as 1.49 events per day. with ``superflares'' approximately 3 times a
year and flares a magntude larger every other year.

Such frequent flaring, as well as compromising the habitability of {\prox} b,
is clearly likely to interfere with measurements of periodicity. It is highly
likely that many of the observations of {\prox} analysed here are affected by
flares.

Together with the rest of the Alpha Centauri system, the age of {\prox} has been
estimated at $5.26 \pm 0.95$ Gyr, slightly older than the Sun \citep{joyce18}.

As shown in Table \ref{table:obstargets}, the bulk of the observations taken
by the Red Dots project are of \prox.

\subsubsection{\bstar}
{\bstar} is, at just under 6ly, the fourth closest star to the solar system after {\prox} and Alpha Centauri A and B. It is of spectral type M4.
It is particularly notable for its very large proper motion of −802.803 mas/yr in Right Ascension and 10,362.542 mas/yr in Declination., the largest of all
stars relative to the solar system. This is combined with a high radial velocity of $-110.6$ km/s \citep{bobylev17}.
The high radial velocity of {\bstar} will bring it close enough
to the solar system to rival or perhaps surpass {\prox} in approximatelyr 10,000 years as estimated by \citet{bobylev10}.
Its rotation period has been progressively given as 130 days in
\citet{benedict98}, 148.6 days in \citet{suarezmascareno15} and 145 $\pm$ 15 days in \citet{toledopadron18}. Activity is low, as noted in the latter.
It has an estimated age of between 7 to 10 Gyr \citep{ribas18}.

\monthonly
\newdate{firstbs}{1}{8}{2017} 

{\bstar} is of particular interest at the moment, following the announcement of
a planet companion, also noted in \citet{ribas18}, and this report has focused
upon it, espacially as the low activity reduces interference from that source
with periodicity calculations.
Of the observations listed in Table \ref{table:filtersnums}, the corresponding observations for {\bstar} are listed in Table \ref{table:bstarsumm}. Observations were taken between August 2017 and
\date{\today}. Due to the high proper motion, care has to be taken to correctly take this into account when identifying it in images.

\begin{table}[!htbp]
\begin{center}
\begin{tabular}{lr} \hline
Filter & Number \\\hline
g & 1,415 \\
i & 1,414 \\
r & 1,414 \\
z & 1,415 \\
H & 2,264 \\
J & 5,557 \\
K & 2,130 \\
\hline
\end{tabular}
\end{center}
\caption{List of observations to date of \bstar. All were made between
\displaydate{firstbs} and \date{\today}}.
\protect\label{table:bstarsumm}
\end{table}
\engdate

\subsubsection{Ross 154}
Ross 154 is 9.6 ly distant, of spectral type M3.5. It has a reasonably high
proper motion of −+637.02 mas/yr in Right Ascension and  –191.64 mas/yr in
Declination and a radial velocity of –10.7 km/s \citep{vanleeuwen07}.

The strong activity is noted in \citet{wargelin08}. Previously, in
\citet{jarrett76}, an activity cycle of about 2 days was reported.

Nowhere reports a rotation period, however a maximum figure for the rotation
period of $3.5 \pm 1.5$ days can be calculated from the rotational
velocity (\vsini) of $5 \pm 1.5$ km/s fiven in \citet{wargelin08} and the radius of 0.24 solar given in
\citet{johnson83}. If the inclination is other than 90\degree the rotation
period will be shorter by a factor of the sine of the inclination angle.
It would be useful to obtain a figure for the rotation period as no planets have
been reported for Ross 154 and a low inclination angle might be one of the
reasons.

\citet{wargelin08} also state that the fast rotation period is indicative of a
n age of less than 1 Gyr.

